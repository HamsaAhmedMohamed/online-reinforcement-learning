\documentclass[11pt,a4paper]{article}

% ---------- Packages ----------
\usepackage[T1]{fontenc}
\usepackage[utf8]{inputenc}
\usepackage[a4paper,margin=2.2cm]{geometry}
\usepackage{microtype}
\usepackage{lmodern}

\usepackage{amsmath,amssymb,amsthm,mathtools}
\usepackage{bm}
\usepackage{physics}        % optional, remove if you dislike it
\usepackage{bbm}            % indicator 1{..}
\usepackage{enumitem}
\usepackage{booktabs}
\usepackage{array}
\usepackage{hyperref}
\usepackage{cleveref}

\usepackage{algorithm}
\usepackage{algpseudocode}

\usepackage{tcolorbox}
\tcbuselibrary{breakable,skins,theorems}

% ---------- Hyperref ----------
\hypersetup{
  colorlinks=true,
  linkcolor=blue,
  urlcolor=blue,
  citecolor=blue,
  pdftitle={Online \& Reinforcement Learning Notes},
}

% ---------- Theorem styles ----------
\theoremstyle{plain}
\newtheorem{theorem}{Theorem}[section]
\newtheorem{lemma}[theorem]{Lemma}
\newtheorem{proposition}[theorem]{Proposition}
\newtheorem{corollary}[theorem]{Corollary}

\theoremstyle{definition}
\newtheorem{definition}[theorem]{Definition}
\newtheorem{assumption}[theorem]{Assumption}

\theoremstyle{remark}
\newtheorem{remark}[theorem]{Remark}

% ---------- Course macros (edit these) ----------
\newcommand{\coursename}{Online and Reinforcement Learning}
\newcommand{\term}{2025--2026}
\newcommand{\authorname}{Hamsa Mohamed}

% ---------- Common RL/OL notation ----------
\newcommand{\R}{\mathbb{R}}
\newcommand{\E}{\mathbb{E}}
\newcommand{\cA}{\mathcal{A}}
\newcommand{\cS}{\mathcal{S}}
\newcommand{\cM}{\mathcal{M}}
\newcommand{\cD}{\mathcal{D}}
\newcommand{\piPolicy}{\pi}
\newcommand{\V}{V}
\newcommand{\Q}{Q}
\newcommand{\Reg}{\mathrm{Reg}}
\newcommand{\KL}{\mathrm{KL}}

% ---------- Nice note boxes ----------
\newtcolorbox{keyidea}{
  breakable,
  colback=blue!4,
  colframe=blue!50!black,
  title=Key idea,
  fonttitle=\bfseries
}

\newtcolorbox{pitfall}{
  breakable,
  colback=red!4,
  colframe=red!55!black,
  title=Pitfall / common mistake,
  fonttitle=\bfseries
}

\newtcolorbox{todoBox}{
  breakable,
  colback=yellow!8,
  colframe=yellow!55!black,
  title=TODO,
  fonttitle=\bfseries
}

\newtcolorbox{proofsketch}{
  breakable,
  colback=green!5,
  colframe=green!40!black,
  title=Proof sketch,
  fonttitle=\bfseries
}

% ---------- Lecture header ----------
\newcommand{\lecture}[3]{%
  \section{Lecture #1: #2}
  \noindent\textbf{Date:} #3 \\
  \textbf{Reading:} \underline{\hspace{0.8\linewidth}} \\
  \textbf{Keywords:} \underline{\hspace{0.78\linewidth}} \\
  \medskip
}

% ---------- Experiment template ----------
\newcommand{\experiment}[2]{%
  \subsection*{Experiment: #1}
  \noindent\textbf{Goal:} #2 \\
  \textbf{Setup:} \\
  \textbf{Metrics:} \\
  \textbf{Results:} \\
  \textbf{Takeaways:} \\
}

% ---------- Title ----------
\title{\coursename\ --- Notes}
\author{\authorname}
\date{\term}

\begin{document}
\maketitle
\tableofcontents
\newpage

% ============================================================
%                HOW TO USE THIS TEMPLATE
% ============================================================
\section*{How to use}
\begin{itemize}[leftmargin=*]
  \item Make one \texttt{\textbackslash lecture\{nr\}\{topic\}\{date\}} per lecture.
  \item Use \texttt{Definition/Theorem/Lemma} environments for clean structure.
  \item Put intuition in \texttt{keyidea} boxes; warnings in \texttt{pitfall}.
  \item For algorithms, use the \texttt{algorithm} environment.
  \item Track open questions in \texttt{todoBox}.
\end{itemize}
\newpage

% ============================================================
%                   LECTURE TEMPLATE
% ============================================================
\lecture{1}{Bandits \& Online Learning Basics}{YYYY-MM-DD}

\subsection{Big picture}
\begin{keyidea}
Online learning = sequential decisions + feedback. Goal often expressed via regret:
\[
\Reg_T = \sum_{t=1}^T \ell_t(w_t) - \min_{w \in \mathcal{W}} \sum_{t=1}^T \ell_t(w).
\]
\end{keyidea}

\subsection{Definitions}
\begin{definition}[Regret]
Define the (pseudo-)regret against a comparator class $\mathcal{W}$ as
\[
\Reg_T = \sum_{t=1}^T \ell_t(w_t) - \min_{w \in \mathcal{W}} \sum_{t=1}^T \ell_t(w).
\]
\end{definition}

\subsection{Key results}
\begin{theorem}[Example theorem statement]
Assume \underline{\hspace{0.7\linewidth}}. Then
\[
\Reg_T \le \underline{\hspace{0.6\linewidth}}.
\]
\end{theorem}

\begin{proofsketch}
Main steps:
\begin{enumerate}[leftmargin=*]
  \item Identify potential / Lyapunov function (e.g., log-sum-exp).
  \item Telescoping sum.
  \item Bound per-round change using convexity / smoothness.
\end{enumerate}
\end{proofsketch}

\subsection{Algorithm}
\begin{algorithm}[H]
\caption{Template Algorithm (fill in)}
\begin{algorithmic}[1]
\State Initialize parameters
\For{$t=1,2,\dots,T$}
  \State Choose action / decision $a_t$ or parameter $w_t$
  \State Observe feedback (loss/reward) and update
\EndFor
\end{algorithmic}
\end{algorithm}

\subsection{Questions}
\begin{todoBox}
\begin{itemize}[leftmargin=*]
  \item Prove lemma X from the lecture notes.
  \item What changes in adversarial vs stochastic setting?
  \item Connection to RL: bandits are MDPs with one state.
\end{itemize}
\end{todoBox}

% ============================================================
%                  RL-SPECIFIC SECTION
% ============================================================
\lecture{2}{MDPs, Value Functions, Bellman Equations}{YYYY-MM-DD}

\subsection{MDP setup}
\begin{definition}[MDP]
An MDP is $(\cS,\cA,P,r,\gamma)$ where
$P(\cdot \mid s,a)$ is the transition kernel and $r(s,a)$ the reward.
\end{definition}

\begin{keyidea}
Value functions:
\[
\V^{\piPolicy}(s) = \E\!\left[\sum_{t=0}^{\infty} \gamma^t r(s_t,a_t)\mid s_0=s, \piPolicy\right],
\quad
\Q^{\piPolicy}(s,a) = \E\!\left[\sum_{t=0}^{\infty} \gamma^t r(s_t,a_t)\mid s_0=s,a_0=a, \piPolicy\right].
\]
\end{keyidea}

\subsection{Bellman equations}
\begin{theorem}[Bellman expectation equation]
For any policy $\piPolicy$,
\[
\V^{\piPolicy}(s) = \sum_{a\in\cA} \piPolicy(a\mid s)\Bigl(r(s,a) + \gamma \E_{s'\sim P(\cdot\mid s,a)}[\V^{\piPolicy}(s')]\Bigr).
\]
\end{theorem}

\begin{pitfall}
It’s easy to mix up \emph{expectation} vs \emph{optimality} Bellman equations.
Write which one you use every time.
\end{pitfall}

% ============================================================
%                 APPENDIX: CHEAT SHEETS
% ============================================================
\appendix
\section{Cheat sheet: common inequalities}
\begin{itemize}[leftmargin=*]
  \item Hoeffding: \underline{\hspace{0.85\linewidth}}
  \item Azuma: \underline{\hspace{0.85\linewidth}}
  \item Pinsker: $\mathrm{TV}(P,Q) \le \sqrt{\frac{1}{2}\KL(P\|Q)}$
\end{itemize}

\section{Cheat sheet: notation}
\begin{tabular}{@{}ll@{}}
\toprule
Symbol & Meaning \\
\midrule
$\Reg_T$ & regret up to time $T$ \\
$\piPolicy$ & policy \\
$\V, \Q$ & value / action-value function \\
$\gamma$ & discount factor \\
\bottomrule
\end{tabular}

\end{document}
